
\documentclass[11pt]{amsart}

%\usepackage{color,graphicx}
%\usepackage{mathrsfs,amsbsy}
\usepackage{ctex}
\usepackage{CJKfntef}
\usepackage{amssymb}
\usepackage{amsmath}
\usepackage{amsfonts}
\usepackage{graphicx}
\usepackage{amsthm}
\usepackage{enumerate}
\usepackage[mathscr]{eucal}
\usepackage{mathrsfs}
\usepackage{verbatim}
\usepackage{listings}
\usepackage{xcolor}
\usepackage{url}
\usepackage{hyperref}
\usepackage{fancybox}

%\usepackage[notcite,notref]{showkeys}

% showkeys  make label explicit on the paper



%python settings
\definecolor{halfgray}{gray}{0.55}
\definecolor{ipython_frame}{RGB}{207, 207, 207}
\definecolor{ipython_bg}{RGB}{247, 247, 247}
\definecolor{ipython_red}{RGB}{186, 33, 33}
\definecolor{ipython_green}{RGB}{0, 128, 0}
\definecolor{ipython_cyan}{RGB}{64, 128, 128}
\definecolor{ipython_purple}{RGB}{170, 34, 255}

\usepackage{listings}
\lstset{
	breaklines=true,
	%
	extendedchars=true,
	literate=
	{á}{{\'a}}1 {é}{{\'e}}1 {í}{{\'i}}1 {ó}{{\'o}}1 {ú}{{\'u}}1
	{Á}{{\'A}}1 {É}{{\'E}}1 {Í}{{\'I}}1 {Ó}{{\'O}}1 {Ú}{{\'U}}1
	{à}{{\`a}}1 {è}{{\`e}}1 {ì}{{\`i}}1 {ò}{{\`o}}1 {ù}{{\`u}}1
	{À}{{\`A}}1 {È}{{\'E}}1 {Ì}{{\`I}}1 {Ò}{{\`O}}1 {Ù}{{\`U}}1
	{ä}{{\"a}}1 {ë}{{\"e}}1 {ï}{{\"i}}1 {ö}{{\"o}}1 {ü}{{\"u}}1
	{Ä}{{\"A}}1 {Ë}{{\"E}}1 {Ï}{{\"I}}1 {Ö}{{\"O}}1 {Ü}{{\"U}}1
	{â}{{\^a}}1 {ê}{{\^e}}1 {î}{{\^i}}1 {ô}{{\^o}}1 {û}{{\^u}}1
	{Â}{{\^A}}1 {Ê}{{\^E}}1 {Î}{{\^I}}1 {Ô}{{\^O}}1 {Û}{{\^U}}1
	{œ}{{\oe}}1 {Œ}{{\OE}}1 {æ}{{\ae}}1 {Æ}{{\AE}}1 {ß}{{\ss}}1
	{ç}{{\c c}}1 {Ç}{{\c C}}1 {ø}{{\o}}1 {å}{{\r a}}1 {Å}{{\r A}}1
	{€}{{\EUR}}1 {£}{{\pounds}}1
}

%%
%% Python definition (c) 1998 Michael Weber
%% Additional definitions (2013) Alexis Dimitriadis
%% modified by me (should not have empty lines)
%%
\lstdefinelanguage{iPython}{
	morekeywords={access,and,break,class,continue,def,del,elif,else,except,exec,finally,for,from,global,if,import,in,is,lambda,not,or,pass,print,raise,return,try,while},%
	%
	% Built-ins
	morekeywords=[2]{abs,all,any,basestring,bin,bool,bytearray,callable,chr,classmethod,cmp,compile,complex,delattr,dict,dir,divmod,enumerate,eval,execfile,file,filter,float,format,frozenset,getattr,globals,hasattr,hash,help,hex,id,input,int,isinstance,issubclass,iter,len,list,locals,long,map,max,memoryview,min,next,object,oct,open,ord,pow,property,range,raw_input,reduce,reload,repr,reversed,round,set,setattr,slice,sorted,staticmethod,str,sum,super,tuple,type,unichr,unicode,vars,xrange,zip,apply,buffer,coerce,intern},%
	%
	sensitive=true,%
	morecomment=[l]\#,%
	morestring=[b]',%
	morestring=[b]",%
	%
	morestring=[s]{'''}{'''},% used for documentation text (mulitiline strings)
	morestring=[s]{"""}{"""},% added by Philipp Matthias Hahn
	%
	morestring=[s]{r'}{'},% `raw' strings
	morestring=[s]{r"}{"},%
	morestring=[s]{r'''}{'''},%
	morestring=[s]{r"""}{"""},%
	morestring=[s]{u'}{'},% unicode strings
	morestring=[s]{u"}{"},%
	morestring=[s]{u'''}{'''},%
	morestring=[s]{u"""}{"""},%
	%
	% {replace}{replacement}{lenght of replace}
	% *{-}{-}{1} will not replace in comments and so on
	literate=
	*{+}{{{\color{ipython_purple}+}}}1
	{-}{{{\color{ipython_purple}-}}}1
	{*}{{{\color{ipython_purple}$^\ast$}}}1
	{/}{{{\color{ipython_purple}/}}}1
	{^}{{{\color{ipython_purple}\^{}}}}1
	{?}{{{\color{ipython_purple}?}}}1
	{!}{{{\color{ipython_purple}!}}}1
	{\%}{{{\color{ipython_purple}\%}}}1
	{<}{{{\color{ipython_purple}<}}}1
	{>}{{{\color{ipython_purple}>}}}1
	{|}{{{\color{ipython_purple}|}}}1
	{\&}{{{\color{ipython_purple}\&}}}1
	{~}{{{\color{ipython_purple}~}}}1
	%
	{==}{{{\color{ipython_purple}==}}}2
	{<=}{{{\color{ipython_purple}<=}}}2
	{>=}{{{\color{ipython_purple}>=}}}2
	%
	{+=}{{{+=}}}2
	{-=}{{{-=}}}2
	{*=}{{{$^\ast$=}}}2
	{/=}{{{/=}}}2,
	%
	literate=
	{á}{{\'a}}1 {é}{{\'e}}1 {í}{{\'i}}1 {ó}{{\'o}}1 {ú}{{\'u}}1
	{Á}{{\'A}}1 {É}{{\'E}}1 {Í}{{\'I}}1 {Ó}{{\'O}}1 {Ú}{{\'U}}1
	{à}{{\`a}}1 {è}{{\`e}}1 {ì}{{\`i}}1 {ò}{{\`o}}1 {ù}{{\`u}}1
	{À}{{\`A}}1 {È}{{\'E}}1 {Ì}{{\`I}}1 {Ò}{{\`O}}1 {Ù}{{\`U}}1
	{ä}{{\"a}}1 {ë}{{\"e}}1 {ï}{{\"i}}1 {ö}{{\"o}}1 {ü}{{\"u}}1
	{Ä}{{\"A}}1 {Ë}{{\"E}}1 {Ï}{{\"I}}1 {Ö}{{\"O}}1 {Ü}{{\"U}}1
	{â}{{\^a}}1 {ê}{{\^e}}1 {î}{{\^i}}1 {ô}{{\^o}}1 {û}{{\^u}}1
	{Â}{{\^A}}1 {Ê}{{\^E}}1 {Î}{{\^I}}1 {Ô}{{\^O}}1 {Û}{{\^U}}1
	{œ}{{\oe}}1 {Œ}{{\OE}}1 {æ}{{\ae}}1 {Æ}{{\AE}}1 {ß}{{\ss}}1
	{ç}{{\c c}}1 {Ç}{{\c C}}1 {ø}{{\o}}1 {å}{{\r a}}1 {Å}{{\r A}}1
	{€}{{\EUR}}1 {£}{{\pounds}}1,
	%
	%   identifierstyle=\color{red}\ttfamily,
	commentstyle=\color{ipython_cyan}\ttfamily,
	stringstyle=\color{ipython_red}\ttfamily,
	keepspaces=true,
	showspaces=false,
	showstringspaces=false,
	%
	rulecolor=\color{ipython_frame},
	frame=single,
	frameround={t}{t}{t}{t},
	framexleftmargin=6mm,
	numbers=left,
	numberstyle=\tiny\color{halfgray},
	%
	%
	backgroundcolor=\color{ipython_bg},
	%   extendedchars=true,
	basicstyle=\scriptsize\ttfamily,
	keywordstyle=\color{ipython_green}\ttfamily,
}


\lstdefinelanguage{Sage}[]{Python}
{morekeywords={False,sage,True},sensitive=true}
\lstset{
	frame=single,%none or single
	showtabs=False,
	showspaces=False,
	showstringspaces=False,
	commentstyle={\ttfamily\color{dgreencolor}},
	keywordstyle={\ttfamily\color{keywordsage}\bfseries},
	stringstyle={\ttfamily\color{dgraycolor}\bfseries},
	language=Sage,
	basicstyle={\fontsize{10pt}{10pt}\ttfamily},
	aboveskip=0.3em,
	belowskip=0.1em,
	numbers=left,
	numberstyle=\footnotesize,
	stepnumber=5,
	breaklines=true,
	backgroundcolor=\color{backsage}
}
\definecolor{dblackcolor}{rgb}{0.0,0.0,0.0}
\definecolor{dbluecolor}{rgb}{0.01,0.02,0.7}
\definecolor{dgreencolor}{rgb}{0.2,0.4,0.0}
\definecolor{dgraycolor}{rgb}{0.30,0.3,0.30}
\newcommand{\dblue}{\color{dbluecolor}\bf}
\newcommand{\dred}{\color{dredcolor}\bf}
\newcommand{\dblack}{\color{dblackcolor}\bf}
\definecolor{backsage}{RGB}{255, 255, 229}
\definecolor{keywordsage}{RGB}{198, 93, 9}




\begin{document}
\date{}

\title
{代码}


\author{周潇翔}
\address{School of Mathematical Sciences\\
University of Science and Technology of China\\
Hefei, 230026\\ P.R. China\\} 
\email{email}





\begin{abstract}
这里总结自己学过的代码供查阅。为啥不用英文?英文的参考文献浩如烟海,也不差我一个啊。对数学系的同学而言,代码的逻辑并不难,大家不会的只是格式而已。第一节横向介绍我们需要啥,之后纵向对每一种语言给出对应的代码。

在这份文档的编译中第一次学python和Github,可以说是紧跟潮流23333
\end{abstract}



\maketitle
%%%%%%%%%%%%%%%%%%%%%%%%%%%%%%%%%%%%%%%%%%%%%%%%%%%%%%%%%%%%%%%%%%%%%%%%%%%%%%%%%%%%%%%%%%%%%


\section{代码需求}

大部分的语言都需要:
\begin{itemize}
	\item 安装+初始代码(Halloworld)
	\item 基本逻辑
	\item 调试
	\item 参考文档
	\item (想保留的)例子
\end{itemize}
以下是具体需求:
\subsection{安装+初始代码}
\begin{itemize}
	\item 简要说明该语言的目的
	\item 说明自己使用何种编译器
	\item 解释该语言的结构(基本框架)
	\item 使用该语言在屏幕中打出"Halloworld"
	\item 必要时给出英文注释
\end{itemize}
\subsection{基本逻辑}
\begin{itemize}
	\item 数据结构类型(数字、字符串、其他结构)
	\item 基本四则运算+mod(若数据结构中包含矩阵,则需要矩阵的各类运算;)
	\item 条件语句
	\item 循环语句
	\item 函数
\end{itemize}
\subsection{调试}
\begin{itemize}
	\item 快捷键
	\begin{itemize}
		\item 运行代码
	\item 注释方式及快捷键(单行注释+多行注释)
	\item 自动补全功能
\item 自动对齐功能
		\item 其他快捷键	
	\end{itemize}
	\item 如何获得帮助
	\item 控制输出	
	\item 如何设置断点
	\item 控制输入


\end{itemize}
\subsection{参考文档}
\begin{itemize}
	\item 官方文档
	\item 民间优秀文档
\end{itemize}


超出科大C语言的知识:编程范式(Programming paradigm)、方法(method)

对于图形化输出的语言,我们还想总结如下知识:
\begin{itemize}
\item 坐标轴:确定位置
\item 基本2D图形:圆、线段、圆弧、钢笔工具
\item 几何计算:距离的计算,交点、中垂线、平行线等
\item 3D作图
\item 背景,控制边界长度
\item 页面布局:盒子模型
\item 数学文本输入:字体
\end{itemize}

\section{python}
\subsection{安装+初始代码}
Python是一门高级的编程语言。他有许多的标准模块(standard module).
\subsection{基本逻辑}
与C语言不同, Python不需要声明变量。当有赋值时不输出结果。

数据类型在\href{https://docs.python.org/3/tutorial/introduction.html#id3}{这里}看到。\footnote{你需要知道那些类型是可改变的(mutable);}

对于数字,Python不仅有int和float,还有Decimal, Fraction and complex numbers这些奇葩的变量。Task:学会Decimal, Fraction.
\begin{lstlisting}[language=iPython]
>>> complex('1+2j')*complex('1+3j')
\end{lstlisting}
四则计算像自然计算一样自然,不过带余除法用\lstinline|//|,余数用\lstinline|%|,幂次用\lstinline|**|.(好符号)\footnote{请小心使用负数的带余除法。}
\begin{lstlisting}[language=iPython]
>>> a,b=8,13 # a++ is not allowed in python
>>> a ** (b-1) % b # verify the Fermat's little theorem
\end{lstlisting}
计算器上的Ans记为\lstinline|_|, \lstinline|round(0.142857,1)|给出$0.1$

python中的逻辑运算符如下:与(\lstinline|and &|),或(\lstinline{or |})

条件语句和循环语句的书写规范详见\href{https://docs.python.org/3/tutorial/controlflow.html}{这里}。以下是计算素数的例子。
\begin{lstlisting}[language=iPython]
import math    # Compute square root
def isPrime(n):    # return true when n is a prime
    for x in range(2, math.isqrt(n)+1):
        if n % x == 0:
            return False;
    else:
        # loop fell through without finding a factor
        return True;
isPrime(57)
\end{lstlisting}
用库可以更加容易地计算素数:
\begin{lstlisting}[language=iPython]
from sympy.ntheory import isprime
isprime(10000019)
\end{lstlisting}
\begin{lstlisting}[language=iPython]
# find primes by sieve method
from sympy import sieve
sieve._reset() # this line for doctest only
# 10000019 in sieve    #10000019 is a prime
sieve.extend(100)
sieve._list
\end{lstlisting}
\subsection{调试}

控制输入使用函数\lstinline|input()|。

\subsection{参考文档}
\href{https://docs.python.org/zh-cn/3/tutorial/index.html}{官方文档}
\href{https://docs.sympy.org/latest/tutorial/index.html#tutorial}{SymPy}
\subsection{(想保留的)例子}
这个例子计算正整数的各位数之和,用到了把数转化为字符串的技巧:
\begin{lstlisting}[language=iPython]
print(sum([int(d) for d in str(int(input("number:")))]))
\end{lstlisting}
这个例子计算$\mathfrak{gl}_n$幂零轨道的维数.
\begin{lstlisting}[language=iPython]
# this computes dimension of orbits of nilpotent elements
young_diagram = [2,1,1,1,1] # Here is the input
n=len(young_diagram)
a,b=0,0
for i in range(n):
    a=a+young_diagram[i]
    for j in range(n):
        if young_diagram[i]<young_diagram[j]:
            b=b+young_diagram[i]
        else:
            b=b+young_diagram[j]
c=a**2 -b
print("the dimension of this orbit is", str(c)+".") # one small trick for printing result
\end{lstlisting}

这个例子想要验证某个猜想。据说是BSD猜想的推论。
\begin{lstlisting}[language=iPython]
from sympy.ntheory import isprime
import math    # Compute square root
def isQuart(k):
    for i in range(k//2):
        if i**4 % k == 3:
            return True
    return False
def solution(m):
    n=math.isqrt(m)+1
    num=0
    for i in range(n):
        for j in range(n):
            for k in range(n):
                if 6* i**2+j**2+18* k**2 == m:
                    #print("find the solution", i, j, k)
                    if i*j*k==0:
                        if k % 2 ==0:
                            num=num+4
                        else:
                            num=num-4
                    else:
                        if k % 2 ==0:
                            num=num+8
                        else:
                            num=num-8                  
    return num
for l in range(25, 10000, 24):
    if isprime(l) & (isQuart(l)==False):
        #print(l,"find the number")
        if solution(l) % 16 ==8:
            print("the conjecture is true for ", l)
        else:
            print("the conjecture is not true for ", l)
\end{lstlisting}
这个例子计算affine quiver正根的可能情形(real positive root + regular simple). 
\begin{lstlisting}[language=iPython]
# Usage of format
# E6 case
orderedCouple1 = [(a1,a2,a3,a4,a5,a6,a7) for a1 in range(2) for a2 in range(3) for a3 in range(4) for a4 in range(3) for a5 in range(2) for a6 in range(3) for a7 in range(2)] 
k=0 # compute the number of positive real root which is possible regular simple
for a1,a2,a3,a4,a5,a6,a7 in orderedCouple1: # do everything in one line*
    if a1**2+a2**2+a3**2+a4**2+a5**2+a6**2+a7**2-a1*a2-a2*a3-a3*a4-a4*a5-a3*a6-a6*a7 ==1:
        k=k+1
        print('                                {0}'.format(a7)) #ugly code have better output
        print('                                {0}'.format(a6))
        print('The positive real root is {0}, {1}, {2}, {3}, {4}.'.format(a1,a2,a3,a4,a5))
        print("")
print('There are {0} results'.format(k))

# E7 case
orderedCouple2 = [(a1,a2,a3,a4,a5,a6,a7,a8) for a1 in range(2) for a2 in range(3) for a3 in range(4) for a4 in range(5) for a5 in range(4) for a6 in range(3) for a7 in range(2) for a8 in range(3)] 
k=0 # compute the number of positive real root which is possible regular simple
for a1,a2,a3,a4,a5,a6,a7,a8 in orderedCouple2: # do everything in one line*
    if a1**2+a2**2+a3**2+a4**2+a5**2+a6**2+a7**2+a8**2-a1*a2-a2*a3-a3*a4-a4*a5-a5*a6-a6*a7-a4*a8 ==1:
        k=k+1
        print('                                   {0}'.format(a8)) #ugly code have better output
        print('The positive real root is {0}, {1}, {2}, {3}, {4}, {5}, {6}.'.format(a1,a2,a3,a4,a5,a6,a7))
        print("")
print('There are {0} results'.format(k))

# E8 case
orderedCouple3 = [(a1,a2,a3,a4,a5,a6,a7,a8,a9) for a1 in range(2) for a2 in range(3) for a3 in range(4) for a4 in range(5) for a5 in range(6) for a6 in range(7) for a7 in range(5) for a8 in range(3) for a9 in range(4)] 
k=0 # compute the number of positive real root which is possible regular simple
for a1,a2,a3,a4,a5,a6,a7,a8,a9 in orderedCouple3: # do everything in one line*
    if a1**2+a2**2+a3**2+a4**2+a5**2+a6**2+a7**2+a8**2+a9**2-a1*a2-a2*a3-a3*a4-a4*a5-a5*a6-a6*a7-a7*a8-a6*a9 ==1:
        k=k+1
        print('                                         {0}'.format(a9)) #ugly code have better output
        print('The positive real root is {0}, {1}, {2}, {3}, {4}, {5}, {6}, {7}.'.format(a1,a2,a3,a4,a5,a6,a7,a8))
        print("")
print('There are {0} results'.format(k))

# E6 case, subspace case
orderedCouple1 = [(a1,a2,a3,a4,a5,a6,a7) for a1 in range(2) for a2 in range(3) for a3 in range(4) for a4 in range(3) for a5 in range(2) for a6 in range(3) for a7 in range(2)] 
k=0 # compute the number of positive real root which is possible regular simple
for a1,a2,a3,a4,a5,a6,a7 in orderedCouple1: # do everything in one line*
    if a1**2+a2**2+a3**2+a4**2+a5**2+a6**2+a7**2-a1*a2-a2*a3-a3*a4-a4*a5-a3*a6-a6*a7 == 1 and a1+a2-3*a3+a4+a5+a6+a7==0:
        k=k+1
        print('                                {0}'.format(a7)) #ugly code have better output
        print('                                {0}'.format(a6))
        print('The positive real root is {0}, {1}, {2}, {3}, {4}.'.format(a1,a2,a3,a4,a5))
        print("")
print('There are {0} results'.format(k))

\end{lstlisting}
\section{sage}
\subsection{安装+初始代码}
Sage是专为数学家设计的程序,基于python,通过集成大量数据库,使用其中的数学函数来简化编程难度。我使用的是网上在线的编辑器\href{https://cocalc.com/projects}{CoCalc},在New中生成Sage worksheet.
你可以在每一行前添加\lstinline|sage: |或者不加。在最开始输入需要的宏包,然后直接进行计算即可。

\begin{lstlisting}[numbers=left,numberstyle=\tiny,numbersep=10pt]
sage: print("Halloworld!233") #The code is the same as python
\end{lstlisting}
\subsection{基本逻辑}
本节的基本内容参见\href{https://www.osgeo.cn/sagemath/thematic_tutorials/tutorial-programming-python.html#tutorial-programming-python}{这里}。
\subsection{参考文档}
\href{https://doc.sagemath.org/}{官方文档}

\href{https://www.osgeo.cn/sagemath/index.html}{Sage V9.1 中文文档}
\subsection{(想保留的)例子}

这个例子能帮助我计算Dynkin quiver的所有不可约表示.
\begin{lstlisting}[numbers=left,numberstyle=\tiny,numbersep=10pt]
sage: Q = DiGraph({1:{2:['a1']},2:{3:['a2']},4:{3:['a3']},5:{4:['a4']},6:{3:['a5']}}).path_semigroup()
sage: M = Q.I(GF(11),3)
sage: M
sage: tauM = M.AR_translate()
sage: tauM
\end{lstlisting}

这个例子能帮助我画出$\Gamma(5)$对应的基本区域.
\begin{lstlisting}[numbers=left,numberstyle=\tiny,numbersep=10pt]
G5 = Gamma(5)
A   = FareySymbol (G5). fundamental_domain ( show_pairing = true )
show( A ,figsize=10 , fontsize=10 )
\end{lstlisting}



\section{latex} 

\subsection{本文参考文献}

 \href{http://blog.sina.com.cn/s/blog_a382a9080102z25i.html}{listings的具体设置}
  
   \href{https://tex.stackexchange.com/questions/84748/fanciest-way-to-include-mathematica-code-in-latex}{fancy版本的Mathematica代码}没学会。
           
\href{https://tex.stackexchange.com/questions/126241/autoindent-in-texmaker}{\LaTeX 中如何自动补全代码?} 不容易。

\href{https://tex.stackexchange.com/questions/235783/listings-recognize-numbers-and-1e-3}{本文python代码格式来源}
\subsection{参考文档}
刘海洋的书:\LaTeX 入门


\section{Mathematica}
\subsection{安装+初始代码}
Mathematica是收费的数学计算软件,参考文档量多但是不够有结构性,导致我的代码往往是临时性的,每次计算都需要重新学习代码。(而且我也没有结构性地保存它们)不过现在可以在这份文档中储存代码了。

\subsection{(想保留的)例子}

这个例子能帮助我计算affine quiver相关的矩阵.
\lstloadlanguages{[11.0]Mathematica}
\begin{lstlisting}[numbers=left,numberstyle=\tiny,numbersep=10pt]
Cp = ({
    {1, 1, 1, 0, 0, 0},
    {0, 1, 1, 0, 0, 0},
    {0, 0, 1, 0, 0, 0},
    {0, 0, 1, 1, 0, 0},
    {0, 0, 1, 1, 1, 0},
    {0, 0, 1, 0, 0, 1}
   });
Ci = Transpose[Cp];
Phi = -Ci.Inverse[Cp]
MatrixPower[Phi, 6]
A = Transpose[Inverse[Ci]] + Inverse[Ci] (*symmetric form*)
\end{lstlisting}
这个例子计算四元数$\mathbb{H}$上的non-reduced norm.
\begin{lstlisting}[numbers=left,numberstyle=\tiny,numbersep=10pt]
Det[({
   {x, -y, -z, -w},
   {y, x, -w, z},
   {z, w, x, -y},
   {w, -z, y, x}
  })]
Factor[w^4 + 2 w^2 x^2 + x^4 + 2 w^2 y^2 + 2 x^2 y^2 + y^4 + 
  2 w^2 z^2 + 2 x^2 z^2 + 2 y^2 z^2 + z^4]
\end{lstlisting}
这个例子画彩色的点.(255位RGB)
\begin{lstlisting}[numbers=left,numberstyle=\tiny,numbersep=10pt]
ListPlot[{Style[{{1, 1}, {2, 3}, {2, 4}}, 
   Interpreter["Color"]["RGB 255 0 0"]], 
  Style[{{1.5, 1}, {2.3, 3}, {2.4, 4}}, 
   Interpreter["Color"]["RGB 0 255 0"]]}]
\end{lstlisting}
\section{电脑快捷键(中级)}
部分参见\href{https://zhuanlan.zhihu.com/p/42281412}{Chrome快捷键}。
\begin{itemize}
	\item  ctrl+win+left:切换桌面
	\item  alt+(shift)+tab:切换任务栏上的程序
	\item  ctrl+(shift)+tab:切换标签页
\end{itemize} 
\section{正则表达式}
正则表达式可以大幅提升搜索和修改代码的效率。
\begin{lstlisting}[numbers=left,numberstyle=\tiny,numbersep=10pt]
.{0,100}(?=;\d{0,3};\d{0,3}m\n);(\d{0,3});(\d{0,3})m\n
\end{lstlisting}
\subsection{参考文档}
推荐\href{https://www.bilibili.com/video/BV19t4y1y7qP}{B站视频}。与之配套的\href{https://codejiaonang.com/#/course/regex_chapter1/}{练习}。

你可以在\href{https://regexr-cn.com/}{regexr}中测试。
\section{热键设置(AutoHotkey)}
这是一款开源软件。为了Hackergame的脚本,同时我猜会大幅度提高打代码的速度。
\subsection{调试}

注释使用分号\lstinline|;|。

\subsection{参考文档}
\href{https://wyagd001.github.io/zh-cn/docs/AutoHotkey.htm}{官方文档}.
;
\subsection{(想保留的)例子}
这个例子模拟鼠标和键盘,使用了条件和循环语句。
\begin{lstlisting}[numbers=left,numberstyle=\tiny,numbersep=10pt]
^j::
Sleep,300
loop, 1
{
    Random, rand, 1, 999
	CoordMode, ToolTip
	Click, 1743 81 1
	Sleep, 100
	Click, 36 64 2
	Sleep, 100	
	Send, {rand}.0.0.0
	Sleep, 100
	Click, 433 323 2
	Sleep, 100
	Send, 0.0.0.0
	Sleep, 100
	Click, 452 321 1
	Sleep, 100		
	if GetKeyState("F10")
	{
	break
	}
	Sleep,1000
}
Return 
\end{lstlisting}
这是第二版本,能够95\%成功提交。
\begin{lstlisting}[numbers=left,numberstyle=\tiny,numbersep=10pt]
#MaxThreadsPerHotkey 3
^j::  ; 热键(可根据您的喜好改变此热键).
#MaxThreadsPerHotkey 1
if KeepWinZRunning  ;  这说明一个潜在的线程正在下面的循环中运行.
{
    KeepWinZRunning := false  ; 向那个线程的循环发出停止的信号.
    return  ; 结束此线程, 这样才可以让下面的线程恢复并得知上一行所做的更改.
}
; 否则:
KeepWinZRunning := true
rand := 0
Loop
{
    ; 以下四行是您要重复的动作(可根据您的需要修改它们):
    MyNumber :=  rand . ".233.233.233"
	Click, 1743 81 1
	Sleep, 70	
	Click, 74 74 2
	Sleep, 70	
	Send, %MyNumber%
	Sleep, 70	
	Click, 429 346 2
	Sleep, 80
	Send, %MyNumber%
	Sleep, 90
	Click, 1073 31 1
	Send ^{Click 432 360 1}
	Sleep, 190
	rand += 1
    ; 但请不要修改下面剩下的内容.
    if not KeepWinZRunning  ; 用户再次按下 Win-Z 来向循环发出停止的信号.
        break  ; 跳出此循环.
}
KeepWinZRunning := false  ; 复位, 为下一次使用热键做准备.
return
\end{lstlisting}
第三个版本可以提交254个,还剩2个。。。
\begin{lstlisting}[numbers=left,numberstyle=\tiny,numbersep=10pt]
#MaxThreadsPerHotkey 3
^j::  
#MaxThreadsPerHotkey 1
if KeepWinZRunning  
{
    KeepWinZRunning := false  
    return  
}
KeepWinZRunning := true
rand := 0
Loop
{
    MyNumber :=  rand . ".233.233.233"
	Click, 1751 91 1
	Sleep, 120
	Click, 74 74 2
	Sleep, 50	
	Send, %MyNumber%
	Sleep, 70	
	Click, 429 346 2
	Sleep, 80
	Send, %MyNumber%
	Sleep, 50
	Click, 587 747 1
	;Click, 1073 31 1
	Send ^{Click 387 437 1}
	Sleep, 60	
	rand += 1
    if not KeepWinZRunning  
        break  
}
KeepWinZRunning := false  
return
\end{lstlisting}
最后成功的版本:
\begin{lstlisting}[numbers=left,numberstyle=\tiny,numbersep=10pt]
#MaxThreadsPerHotkey 3
^j::  
#MaxThreadsPerHotkey 1
if KeepWinZRunning  
{
    KeepWinZRunning := false  
    return  
}
KeepWinZRunning := true
rand := 0
Loop
{
    MyNumber :=  rand . ".233.233.233"
	Click, 1751 91 1
	Sleep, 120
	Click, 74 74 2
	Sleep, 50	
	Send, %MyNumber%
	Sleep, 70	
	Click, 429 346 2
	Sleep, 80
	Send, %MyNumber%
	Sleep, 40
	Click, 587 747 1
	;Click, 1073 31 1
	Send ^{Click 387 437 1}
	Sleep, 60	
	rand += 1
    if not KeepWinZRunning  
        break  
}
KeepWinZRunning := false  
return
\end{lstlisting}
\subsection{Tasks}
通过该软件创建一个\LaTeX 的脚本,要求:
\begin{itemize}
	\item 自动翻译iff, SES, LES, mfld, rep...
	\item 长段代码如equ+aligned
	\item 只在打开TeXstudio时生效
\end{itemize}
通过该软件创建一个脚本,要求:
\begin{itemize}
	\item 自动打开日常软件
	\item 将其放入开机启动项
\end{itemize}

%%%%%%%%%%%%%%%%%%%%%%%%%%%%%%%%%%%%%%%%%%%%%%%%%%%%%%%%%%%%%%%%%%%%%%%%%%%%%%%%%%%%%%%%%%%%%

 
   


%%%%%%%%%%%%%%%%%%%%%%%%%%%%%%%%%%%%%%%%%%%%%%%%%%%%%%%%%%%%%%%%%%%%%%%%%%

 




%%%%%%%%%%%%%%%%%%%%%%%%%%%%%%%%%%%%%%%%%%%%%%%%%%%%%%%%%%%%%%%%%%%%%%%%%%%%%%%%%%%%%%%%%%%%%%%




\begin{thebibliography}{99}

 
%\bibitem{AF12}%
%Antunes, P., Freitas, P.: Optimal spectral rectangles and lattice ellipses. \emph{Proc. Royal Soc. London Ser. A.} \textbf{469} (2012), 20120492.


  

\end{thebibliography}


\end{document}




