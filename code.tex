
\documentclass[11pt]{amsart}

%\usepackage{color,graphicx}
%\usepackage{mathrsfs,amsbsy}
\usepackage{ctex}
\usepackage{CJKfntef}
\usepackage{amssymb}
\usepackage{amsmath}
\usepackage{amsfonts}
\usepackage{graphicx}
\usepackage{amsthm}
\usepackage{enumerate}
\usepackage[mathscr]{eucal}
\usepackage{mathrsfs}
\usepackage{verbatim}
\usepackage{listings}
\usepackage{xcolor}
\usepackage{url}
\usepackage{hyperref}
\usepackage{fancybox}

%\usepackage[notcite,notref]{showkeys}

% showkeys  make label explicit on the paper



%python settings
\definecolor{halfgray}{gray}{0.55}
\definecolor{ipython_frame}{RGB}{207, 207, 207}
\definecolor{ipython_bg}{RGB}{247, 247, 247}
\definecolor{ipython_red}{RGB}{186, 33, 33}
\definecolor{ipython_green}{RGB}{0, 128, 0}
\definecolor{ipython_cyan}{RGB}{64, 128, 128}
\definecolor{ipython_purple}{RGB}{170, 34, 255}

\usepackage{listings}
\lstset{
	breaklines=true,
	%
	extendedchars=true,
	literate=
	{á}{{\'a}}1 {é}{{\'e}}1 {í}{{\'i}}1 {ó}{{\'o}}1 {ú}{{\'u}}1
	{Á}{{\'A}}1 {É}{{\'E}}1 {Í}{{\'I}}1 {Ó}{{\'O}}1 {Ú}{{\'U}}1
	{à}{{\`a}}1 {è}{{\`e}}1 {ì}{{\`i}}1 {ò}{{\`o}}1 {ù}{{\`u}}1
	{À}{{\`A}}1 {È}{{\'E}}1 {Ì}{{\`I}}1 {Ò}{{\`O}}1 {Ù}{{\`U}}1
	{ä}{{\"a}}1 {ë}{{\"e}}1 {ï}{{\"i}}1 {ö}{{\"o}}1 {ü}{{\"u}}1
	{Ä}{{\"A}}1 {Ë}{{\"E}}1 {Ï}{{\"I}}1 {Ö}{{\"O}}1 {Ü}{{\"U}}1
	{â}{{\^a}}1 {ê}{{\^e}}1 {î}{{\^i}}1 {ô}{{\^o}}1 {û}{{\^u}}1
	{Â}{{\^A}}1 {Ê}{{\^E}}1 {Î}{{\^I}}1 {Ô}{{\^O}}1 {Û}{{\^U}}1
	{œ}{{\oe}}1 {Œ}{{\OE}}1 {æ}{{\ae}}1 {Æ}{{\AE}}1 {ß}{{\ss}}1
	{ç}{{\c c}}1 {Ç}{{\c C}}1 {ø}{{\o}}1 {å}{{\r a}}1 {Å}{{\r A}}1
	{€}{{\EUR}}1 {£}{{\pounds}}1
}

%%
%% Python definition (c) 1998 Michael Weber
%% Additional definitions (2013) Alexis Dimitriadis
%% modified by me (should not have empty lines)
%%
\lstdefinelanguage{iPython}{
	morekeywords={access,and,break,class,continue,def,del,elif,else,except,exec,finally,for,from,global,if,import,in,is,lambda,not,or,pass,print,raise,return,try,while},%
	%
	% Built-ins
	morekeywords=[2]{abs,all,any,basestring,bin,bool,bytearray,callable,chr,classmethod,cmp,compile,complex,delattr,dict,dir,divmod,enumerate,eval,execfile,file,filter,float,format,frozenset,getattr,globals,hasattr,hash,help,hex,id,input,int,isinstance,issubclass,iter,len,list,locals,long,map,max,memoryview,min,next,object,oct,open,ord,pow,property,range,raw_input,reduce,reload,repr,reversed,round,set,setattr,slice,sorted,staticmethod,str,sum,super,tuple,type,unichr,unicode,vars,xrange,zip,apply,buffer,coerce,intern},%
	%
	sensitive=true,%
	morecomment=[l]\#,%
	morestring=[b]',%
	morestring=[b]",%
	%
	morestring=[s]{'''}{'''},% used for documentation text (mulitiline strings)
	morestring=[s]{"""}{"""},% added by Philipp Matthias Hahn
	%
	morestring=[s]{r'}{'},% `raw' strings
	morestring=[s]{r"}{"},%
	morestring=[s]{r'''}{'''},%
	morestring=[s]{r"""}{"""},%
	morestring=[s]{u'}{'},% unicode strings
	morestring=[s]{u"}{"},%
	morestring=[s]{u'''}{'''},%
	morestring=[s]{u"""}{"""},%
	%
	% {replace}{replacement}{lenght of replace}
	% *{-}{-}{1} will not replace in comments and so on
	literate=
	*{+}{{{\color{ipython_purple}+}}}1
	{-}{{{\color{ipython_purple}-}}}1
	{*}{{{\color{ipython_purple}$^\ast$}}}1
	{/}{{{\color{ipython_purple}/}}}1
	{^}{{{\color{ipython_purple}\^{}}}}1
	{?}{{{\color{ipython_purple}?}}}1
	{!}{{{\color{ipython_purple}!}}}1
	{\%}{{{\color{ipython_purple}\%}}}1
	{<}{{{\color{ipython_purple}<}}}1
	{>}{{{\color{ipython_purple}>}}}1
	{|}{{{\color{ipython_purple}|}}}1
	{\&}{{{\color{ipython_purple}\&}}}1
	{~}{{{\color{ipython_purple}~}}}1
	%
	{==}{{{\color{ipython_purple}==}}}2
	{<=}{{{\color{ipython_purple}<=}}}2
	{>=}{{{\color{ipython_purple}>=}}}2
	%
	{+=}{{{+=}}}2
	{-=}{{{-=}}}2
	{*=}{{{$^\ast$=}}}2
	{/=}{{{/=}}}2,
	%
	literate=
	{á}{{\'a}}1 {é}{{\'e}}1 {í}{{\'i}}1 {ó}{{\'o}}1 {ú}{{\'u}}1
	{Á}{{\'A}}1 {É}{{\'E}}1 {Í}{{\'I}}1 {Ó}{{\'O}}1 {Ú}{{\'U}}1
	{à}{{\`a}}1 {è}{{\`e}}1 {ì}{{\`i}}1 {ò}{{\`o}}1 {ù}{{\`u}}1
	{À}{{\`A}}1 {È}{{\'E}}1 {Ì}{{\`I}}1 {Ò}{{\`O}}1 {Ù}{{\`U}}1
	{ä}{{\"a}}1 {ë}{{\"e}}1 {ï}{{\"i}}1 {ö}{{\"o}}1 {ü}{{\"u}}1
	{Ä}{{\"A}}1 {Ë}{{\"E}}1 {Ï}{{\"I}}1 {Ö}{{\"O}}1 {Ü}{{\"U}}1
	{â}{{\^a}}1 {ê}{{\^e}}1 {î}{{\^i}}1 {ô}{{\^o}}1 {û}{{\^u}}1
	{Â}{{\^A}}1 {Ê}{{\^E}}1 {Î}{{\^I}}1 {Ô}{{\^O}}1 {Û}{{\^U}}1
	{œ}{{\oe}}1 {Œ}{{\OE}}1 {æ}{{\ae}}1 {Æ}{{\AE}}1 {ß}{{\ss}}1
	{ç}{{\c c}}1 {Ç}{{\c C}}1 {ø}{{\o}}1 {å}{{\r a}}1 {Å}{{\r A}}1
	{€}{{\EUR}}1 {£}{{\pounds}}1,
	%
	%   identifierstyle=\color{red}\ttfamily,
	commentstyle=\color{ipython_cyan}\ttfamily,
	stringstyle=\color{ipython_red}\ttfamily,
	keepspaces=true,
	showspaces=false,
	showstringspaces=false,
	%
	rulecolor=\color{ipython_frame},
	frame=single,
	frameround={t}{t}{t}{t},
	framexleftmargin=6mm,
	numbers=left,
	numberstyle=\tiny\color{halfgray},
	%
	%
	backgroundcolor=\color{ipython_bg},
	%   extendedchars=true,
	basicstyle=\scriptsize\ttfamily,
	keywordstyle=\color{ipython_green}\ttfamily,
}


\lstdefinelanguage{Sage}[]{Python}
{morekeywords={False,sage,True},sensitive=true}
\lstset{
	frame=single,%none or single
	showtabs=False,
	showspaces=False,
	showstringspaces=False,
	commentstyle={\ttfamily\color{dgreencolor}},
	keywordstyle={\ttfamily\color{keywordsage}\bfseries},
	stringstyle={\ttfamily\color{dgraycolor}\bfseries},
	language=Sage,
	basicstyle={\fontsize{10pt}{10pt}\ttfamily},
	aboveskip=0.3em,
	belowskip=0.1em,
	numbers=left,
	numberstyle=\footnotesize,
	stepnumber=5,
	breaklines=true,
	backgroundcolor=\color{backsage}
}
\definecolor{dblackcolor}{rgb}{0.0,0.0,0.0}
\definecolor{dbluecolor}{rgb}{0.01,0.02,0.7}
\definecolor{dgreencolor}{rgb}{0.2,0.4,0.0}
\definecolor{dgraycolor}{rgb}{0.30,0.3,0.30}
\newcommand{\dblue}{\color{dbluecolor}\bf}
\newcommand{\dred}{\color{dredcolor}\bf}
\newcommand{\dblack}{\color{dblackcolor}\bf}
\definecolor{backsage}{RGB}{255, 255, 229}
\definecolor{keywordsage}{RGB}{198, 93, 9}




\begin{document}
\date{}

\title
{代码}


\author{周潇翔}
\address{School of Mathematical Sciences\\
University of Science and Technology of China\\
Hefei, 230026\\ P.R. China\\} 
\email{email}





\begin{abstract}
这里总结自己学过的代码供查阅。为啥不用英文?英文的参考文献浩如烟海,也不差我一个啊。对数学系的同学而言,代码的逻辑并不难,大家不会的只是格式而已。第一节横向介绍我们需要啥,之后纵向对每一种语言给出对应的代码。
\end{abstract}



\maketitle
%%%%%%%%%%%%%%%%%%%%%%%%%%%%%%%%%%%%%%%%%%%%%%%%%%%%%%%%%%%%%%%%%%%%%%%%%%%%%%%%%%%%%%%%%%%%%


\section{代码需求}

大部分的语言都需要:
\begin{itemize}
	\item 安装+初始代码(Halloworld)
	\item 基本逻辑
	\item 调试
	\item 参考文档
	\item (想保留的)例子
\end{itemize}
以下是具体需求:
\subsection{安装+初始代码}
\begin{itemize}
	\item 简要说明该语言的目的
	\item 说明自己使用何种编译器
	\item 解释该语言的结构(基本框架)
	\item 使用该语言在屏幕中打出"Halloworld"
	\item 必要时给出英文注释
\end{itemize}
\subsection{基本逻辑}
\begin{itemize}
	\item 数据结构类型(数字、字符串、其他结构)
	\item 基本四则运算+mod(若数据结构中包含矩阵,则需要矩阵的各类运算;)
	\item 条件语句
	\item 循环语句
	\item 函数
\end{itemize}
\subsection{调试}
\begin{itemize}
	\item 快捷键
	\begin{itemize}
		\item 运行代码
	\item 注释方式及快捷键(单行注释+多行注释)
	\item 自动补全功能
\item 自动对齐功能
		\item 其他快捷键	
	\end{itemize}
	\item 如何获得帮助
	\item 控制输出	
	\item 如何设置断点
	\item 控制输入


\end{itemize}
\subsection{参考文档}
\begin{itemize}
	\item 官方文档
	\item 民间优秀文档
\end{itemize}


超出科大C语言的知识:编程范式(Programming paradigm)、方法(method)

\section{python}
\subsection{安装+初始代码}
Python是一门高级的编程语言。他有许多的标准模块(standard module).
\subsection{基本逻辑}
与C语言不同, Python不需要声明变量。当有赋值时不输出结果。

数据类型在\href{https://docs.python.org/3/tutorial/introduction.html#id3}{这里}看到。\footnote{你需要知道那些类型是可改变的(mutable);}

对于数字,Python不仅有int和float,还有Decimal, Fraction and complex numbers这些奇葩的变量。Task:学会Decimal, Fraction.
\begin{lstlisting}[language=iPython]
>>> complex('1+2j')*complex('1+3j')
\end{lstlisting}
四则计算像自然计算一样自然,不过带余除法用\lstinline|//|,余数用\lstinline|%|,幂次用\lstinline|**|.(好符号)\footnote{请小心使用负数的带余除法。}
\begin{lstlisting}[language=iPython]
>>> a,b=8,13 # a++ is not allowed in python
>>> a ** (b-1) % b # verify the Fermat's little theorem
\end{lstlisting}
计算器上的Ans记为\lstinline|_|, \lstinline|round(0.142857,1)|给出$0.1$

条件语句和循环语句的书写规范详见\href{https://docs.python.org/3/tutorial/controlflow.html}{这里}。以下是计算素数的例子。
\begin{lstlisting}[language=iPython]
>>> for n in range(2, 100):
...     for x in range(2, n):
...         if n % x == 0:
...             break
...     else:
...         # loop fell through without finding a factor
...         print(n, end=",")
\end{lstlisting}

\subsection{调试}

控制输入使用函数\lstinline|input()|。

\subsection{参考文档}
\href{https://docs.python.org/zh-cn/3/tutorial/index.html}{官方文档}




\section{sage}
\subsection{安装+初始代码}
Sage是专为数学家设计的程序,基于python,通过集成大量数据库,使用其中的数学函数来简化编程难度。我使用的是网上在线的编辑器\href{https://cocalc.com/projects}{CoCalc},在New中生成Sage worksheet.
你可以在每一行前添加\lstinline|sage: |或者不加。在最开始输入需要的宏包,然后直接进行计算即可。

\begin{lstlisting}[numbers=left,numberstyle=\tiny,numbersep=10pt]
sage: print("Halloworld!233") #The code is the same as python
\end{lstlisting}
\subsection{基本逻辑}
本节的基本内容参见\href{https://www.osgeo.cn/sagemath/thematic_tutorials/tutorial-programming-python.html#tutorial-programming-python}{这里}。
\subsection{参考文档}
\href{https://doc.sagemath.org/}{官方文档}

\href{https://www.osgeo.cn/sagemath/index.html}{Sage V9.1 中文文档}
\subsection{(想保留的)例子}

这个例子能帮助我计算Dynkin quiver的所有不可约表示.
\begin{lstlisting}[numbers=left,numberstyle=\tiny,numbersep=10pt]
sage: Q = DiGraph({1:{2:['a1']},2:{3:['a2']},4:{3:['a3']},5:{4:['a4']},6:{3:['a5']}}).path_semigroup()
sage: M = Q.I(GF(11),3)
sage: M
sage: tauM = M.AR_translate()
sage: tauM
\end{lstlisting}





\section{latex} 

\subsection{本文参考文献}

 \href{http://blog.sina.com.cn/s/blog_a382a9080102z25i.html}{listings的具体设置}
           
\href{https://tex.stackexchange.com/questions/126241/autoindent-in-texmaker}{\LaTeX 中如何自动补全代码?} 不容易。

\href{https://tex.stackexchange.com/questions/235783/listings-recognize-numbers-and-1e-3}{本文python代码格式来源}
\subsection{参考文档}
刘海洋的书:\LaTeX 入门
\section{电脑快捷键(中级)}
部分参见\href{https://zhuanlan.zhihu.com/p/42281412}{Chrome快捷键}。
\begin{itemize}
	\item  ctrl+win+left:切换桌面
	\item  alt+(shift)+tab:切换任务栏上的程序
	\item  ctrl+(shift)+tab:切换标签页
\end{itemize} 



%%%%%%%%%%%%%%%%%%%%%%%%%%%%%%%%%%%%%%%%%%%%%%%%%%%%%%%%%%%%%%%%%%%%%%%%%%%%%%%%%%%%%%%%%%%%%

 
   


%%%%%%%%%%%%%%%%%%%%%%%%%%%%%%%%%%%%%%%%%%%%%%%%%%%%%%%%%%%%%%%%%%%%%%%%%%

 




%%%%%%%%%%%%%%%%%%%%%%%%%%%%%%%%%%%%%%%%%%%%%%%%%%%%%%%%%%%%%%%%%%%%%%%%%%%%%%%%%%%%%%%%%%%%%%%




\begin{thebibliography}{99}

 
%\bibitem{AF12}%
%Antunes, P., Freitas, P.: Optimal spectral rectangles and lattice ellipses. \emph{Proc. Royal Soc. London Ser. A.} \textbf{469} (2012), 20120492.


  

\end{thebibliography}


\end{document}




