
\documentclass{amsart}
%Typical documenttypes: article/book
%some examples:
%\documentclass[reqno,11pt]{book}   %%%for books
%\documentclass[]{minimal}			%%%for Minimal Working Example


%for beamers, you have to change a lot. Especially, remove the package enumitem!!!



%%%%%%%%%%%%%%%%%%%% setting for fast compiling

%\special{dvipdfmx:config z 0}		% no compression

\includeonly{chapters/chapter9}		% In practice, use an empty document called "chapter9"	% usually for printing books






%%%%%%%%%%%%%%%%%%%% here we include packages

%%%basic packages for math articles
\usepackage{amssymb}
\usepackage{amsthm}
\usepackage{amsmath}
\usepackage{amsfonts}
\usepackage[shortlabels]{enumitem}	% It supersedes both enumerate and mdwlist. The package option shortlabels is included to configure the labels like in enumerate.

%%%packages for special symbols
\usepackage{pifont}					% Access to PostScript standard Symbol and Dingbats fonts
\usepackage{wasysym}				% additional characters
\usepackage{bm}						% bold fonts: \bm{...}
\usepackage{extarrows}				% may be replaced by tikz-cd
%\usepackage{unicode-math}			% unicode maths for math fonts, now I don't know how to include it

%%%basic packages for fancy electronic documents
\usepackage[colorlinks]{hyperref}
\usepackage[table,hyperref]{xcolor} 			% before tikz-cd. 
%\usepackage[table,hyperref,monochrome]{xcolor}	% disable colored output (black and white)

%%%packages for figures and tables (general setting)
\usepackage{float}				%Improved interface for floating objects
\usepackage{caption,subcaption}
\usepackage{adjustbox}			% for me it is usually used in tables 
\usepackage{stackengine}		%baseline changes

%%%packages for commutative diagrams
\usepackage{tikz-cd}
%\usepackage{quiver}			% see https://q.uiver.app/

%%%packages for pictures
\usepackage[width=0.5,tiewidth=0.7]{strands}
\usepackage{graphicx}			% Enhanced support for graphics

%%%packages for tables and general settings
\usepackage{array}
\usepackage{makecell}
\usepackage{multicol}
\usepackage{multirow}
\usepackage{diagbox}
\usepackage{longtable}

%%%packages for ToC, LoF and LoT







 %https://tex.stackexchange.com/questions/58852/possible-incompatibility-with-enumitem










%%%%%%%%%%%%%%%%%%%% here we include theoremstyles

\numberwithin{equation}{section}

\theoremstyle{plain}
\newtheorem{theorem}{Theorem}[section]

\newtheorem{setting}[theorem]{Setting}
\newtheorem{definition}[theorem]{Definition}
\newtheorem{lemma}[theorem]{Lemma}
\newtheorem{proposition}[theorem]{Proposition}
\newtheorem{corollary}[theorem]{Corollary}
\newtheorem{conjecture}[theorem]{Conjecture}

\newtheorem{claim}[theorem]{Claim}
\newtheorem{eg}[theorem]{Example}
\newtheorem{ex}[theorem]{Exercise}
\newtheorem{fact}[theorem]{Fact}
\newtheorem{ques}[theorem]{Question}
\newtheorem{warning}[theorem]{Warning}



\newtheorem*{bbox}{Black box}
\newtheorem*{notation}{Conventions and Notations}


\numberwithin{equation}{section}


\theoremstyle{remark}

\newtheorem{remark}[theorem]{Remark}
\newtheorem*{remarks}{Remarks}

%%% for important theorems
%\newtheoremstyle{theoremletter}{4mm}{1mm}{\itshape}{ }{\bfseries}{}{ }{}
%\theoremstyle{theoremletter}
%\newtheorem{theoremA}{Theorem}
%\renewcommand{\thetheoremA}{A}
%\newtheorem{theoremB}{Theorem}
%\renewcommand{\thetheoremB}{B}







%%%%%%%%%%%%%%%%%%%% here we declare some symbols

%%%%%%%DeclareMathOperator
%see here for why newcommand is better for DeclareMathOperator: https://tex.stackexchange.com/questions/67506/newcommand-vs-declaremathoperator

%%%%%basic symbols. Keep them!

%%%symbols for sets and maps
\DeclareMathOperator{\pt}{\operatorname{pt}}	%points. Other possibilities are \{pt\}, \{*\}, pt, * ...
\DeclareMathOperator{\Id}{\operatorname{Id}}	%identity in groups.
\DeclareMathOperator{\Img}{\operatorname{Im}}

\DeclareMathOperator{\Ob}{\operatorname{Ob}}
\DeclareMathOperator{\Mor}{\operatorname{Mor}}	%difference of Mor and Hom: Hom is usually for abelian categories
\DeclareMathOperator{\Hom}{\operatorname{Hom}}	\DeclareMathOperator{\End}{\operatorname{End}}
\DeclareMathOperator{\Aut}{\operatorname{Aut}}

%%%symbols for linear algebras and 
%%linear algebras
\DeclareMathOperator{\tr}{\operatorname{tr}}
\DeclareMathOperator{\diag}{\operatorname{diag}}	%for diagonal matrices

%%abstract algebras
\DeclareMathOperator{\ord}{\operatorname{ord}}
\DeclareMathOperator{\gr}{\operatorname{gr}}
\DeclareMathOperator{\Frac}{\operatorname{Frac}}

%%%symbols for basic geometries
\DeclareMathOperator{\vol}{\operatorname{vol}}	%volume
\DeclareMathOperator{\dist}{\operatorname{dist}}
\DeclareMathOperator{\supp}{\operatorname{supp}}

%%%symbols for category
%%names of categories
\DeclareMathOperator{\Mod}{\operatorname{Mod}}
\DeclareMathOperator{\Vect}{\operatorname{Vect}}
\DeclareMathOperator{\rep}{\operatorname{rep}} %usually rep means the category of finite dimensional representations, while Rep means the category of representations.
\DeclareMathOperator{\Rep}{\operatorname{Rep}}


%%%symbols for homological algebras
\DeclareMathOperator{\Tor}{\operatorname{Tor}}
\DeclareMathOperator{\Ext}{\operatorname{Ext}}
\DeclareMathOperator{\gldim}{\operatorname{gl.dim}}
\DeclareMathOperator{\projdim}{\operatorname{proj.dim}}
\DeclareMathOperator{\injdim}{\operatorname{inj.dim}}
\DeclareMathOperator{\rad}{\operatorname{rad}}


%%%symbols for algebraic groups
\DeclareMathOperator{\GL}{\operatorname{GL}}
\DeclareMathOperator{\SL}{\operatorname{SL}}

%%%symbols for typical varieties
\DeclareMathOperator{\Gr}{\operatorname{Gr}}
\DeclareMathOperator{\Flag}{\operatorname{Flag}}

%%%symbols for basic algebraic geometry
\DeclareMathOperator{\Spec}{\operatorname{Spec}}
\DeclareMathOperator{\Coh}{\operatorname{Coh}}
\newcommand{\Dcoh}{\mathcal{D}_{\operatorname{Coh}}}%%%This one shows the difference between \DeclareMathOperator and \newcommand
\DeclareMathOperator{\Pic}{\operatorname{Pic}}
\DeclareMathOperator{\Jac}{\operatorname{Jac}}

%%%%%advanced symbols. Choose the part you need!

%%%symbols for algebraic representation theory
\DeclareMathOperator{\Irr}{\operatorname{Irr}}
\DeclareMathOperator{\ind}{\operatorname{ind}}	%\ind(Q) means the set of  equivalence classes of finite dimensional indecomposable representations
\DeclareMathOperator{\Res}{\operatorname{Res}}
\DeclareMathOperator{\Ind}{\operatorname{Ind}}
\DeclareMathOperator{\cInd}{\operatorname{c-Ind}}


%%%symbols for algebraic topology
\DeclareMathOperator{\EGG}{\operatorname{E}\!}
\DeclareMathOperator{\BGG}{\operatorname{B}\!}

\DeclareMathOperator{\chern}{\operatorname{ch}^{*}}
\DeclareMathOperator{\Td}{\operatorname{Td}}
\DeclareMathOperator{\AS}{\operatorname{AS}}	%Atiyah--Segal completion theorem 

%%%symbols for Auslander--Reiten theory 
\DeclareMathOperator{\Modup}{\overline{\operatorname{mod}}}
\DeclareMathOperator{\Moddown}{\underline{\operatorname{mod}}}
\DeclareMathOperator{\Homup}{\overline{\operatorname{Hom}}}
\DeclareMathOperator{\Homdown}{\underline{\operatorname{Hom}}}


%%%symbols for operad
\DeclareMathOperator{\Com}{\operatorname{\mathcal{C}om}}
\DeclareMathOperator{\Ass}{\operatorname{\mathcal{A}ss}}
\DeclareMathOperator{\Lie}{\operatorname{\mathcal{L}ie}}
\DeclareMathOperator{\calEnd}{\operatorname{\mathcal{E}nd}} %cal=\mathcal


%%%%%personal symbols. Use at your own risk!

%%%symbols only for master thesis
\DeclareMathOperator{\ptt}{\operatorname{par}}	%the partition map
\DeclareMathOperator{\str}{\operatorname{str}}	%strict case
\DeclareMathOperator{\RRep}{\widetilde{\operatorname{Rep}}}
\DeclareMathOperator{\Rpt}{\operatorname{R}}
\DeclareMathOperator{\Rptc}{\operatorname{\mathcal{R}}}
\DeclareMathOperator{\Spt}{\operatorname{S}}
\DeclareMathOperator{\Sptc}{\operatorname{\mathcal{S}}}
\DeclareMathOperator{\Kcurl}{\operatorname{\mathcal{K}}}
\DeclareMathOperator{\Hcurl}{\operatorname{\mathcal{H}}}
\DeclareMathOperator{\eu}{\operatorname{eu}}
\DeclareMathOperator{\Eu}{\operatorname{Eu}}
\DeclareMathOperator{\dimv}{\operatorname{\underline{\mathbf{dim}}}}
\DeclareMathOperator{\St}{\mathcal{Z}}

%%%%%symbols which haven't been classified. Add your own math operators here!


\DeclareMathOperator{\Modr}{\operatorname{-Mod}}





%%%%%%%newcommand

%%%basic symbols
\newcommand{\norm}[1]{\Vert{#1}\Vert}

%%%symbols only for master thesis
\newcommand{\dimvec}[1]{\mathbf{#1}}
\newcommand{\abdimvec}[1]{|\dimvec{#1}|}
\newcommand{\ftdimvec}[1]{\underline{\dimvec{#1}}}

\newcommand{\absgp}[1]{\mathbb{#1}}
\newcommand{\WWd}{\absgp{W}_{\abdimvec{d}}}
\newcommand{\Wd}{W_{\dimvec{d}}}
\newcommand{\MinWd}{\operatorname{Min}(\absgp{W}_{\abdimvec{d}},W_{\dimvec{d}})}
\newcommand{\Compd}{\operatorname{Comp}_{\dimvec{d}}}
\newcommand{\Shuffled}{\operatorname{Shuffle}_{\dimvec{d}}}

\newcommand{\Omcell}{\Omega}
\newcommand{\OOmcell}{\boldsymbol{\Omega}}
\newcommand{\Vcell}{\mathcal{V}}
\newcommand{\VVcell}{\boldsymbol{\mathcal{V}}}
\newcommand{\Ocell}{\mathcal{O}}
\newcommand{\OOcell}{\boldsymbol{\mathcal{O}}}
\newcommand{\preimage}[1]{\widetilde{#1}}
\newcommand{\orde}{\operatorname{ord}_e}
\newcommand{\fakestar}{*}

%as the subscription of Hom
\newcommand{\Alggp}{\text{-Alg gp}}







%%%%%%%%%%%%%%%%%%%% here we make some blocks for special features. 

%%%% todo notes %%%%
\usepackage[colorinlistoftodos,textsize=footnotesize]{todonotes}
\setlength{\marginparwidth}{2.5cm}
\newcommand{\leftnote}[1]{\reversemarginpar\marginnote{\footnotesize #1}}
\newcommand{\rightnote}[1]{\normalmarginpar\marginnote{\footnotesize #1}\reversemarginpar}









%%%%%%%%%%%%%%%%%%%% here we make some global settings. Understand everything here before you make a document!

\usepackage[a4paper,left=3cm,right=3cm,bottom=4cm]{geometry}
\usepackage{indentfirst}	% Indent first paragraph after section header

\setcounter{tocdepth}{2}


%https://latexref.xyz/_005cparindent-_0026-_005cparskip.html
\setlength{\parindent}{15pt}	
\setlength{\parskip}{0pt plus1pt}

%\setlength\intextsep{0cm}
%\setlength\textfloatsep{0cm}
\def\arraystretch{1.2}
%\setcounter{secnumdepth}{3}

\allowdisplaybreaks


\begin{document}

% The beginning depends on the documentclass. Rewrite this part if you use different documentclass!
\date{\today}

\title
{Mathematical Abbreviations
}
\author{Xiaoxiang Zhou}
\address{School of Mathematical Sciences\\
University of Bonn\\
Bonn, 53115\\ Germany\\} 
\email{email:xx352229@mail.ustc.edu.cn}


\maketitle
\tableofcontents


\section{Introduction}
This is a document for mathematical abbreviations. See \href{https://en.wikipedia.org/wiki/List_of_mathematical_abbreviations}{wiki} for a more completed description. See also the mathematical \href{https://en.wikipedia.org/wiki/List_of_mathematical_jargon}{jargons}.

\section{Jargons}

\begin{longtable}{l|l}
\hline
c.f. & compare (as a reference) \\ \hline
WLOG & without loss of generality  \\ \hline
 & \\ \hline
 & \\ \hline
 & \\ \hline
 & \\ \hline
 & \\ \hline
\end{longtable}

\section{Mathematicians}

%https://www.bejson.com/othertools/stringarraysort/
\begin{remark}
Białynicki-Birula and Levi-Civita are just single persons.
\end{remark}

\begin{longtable}{l|l}
\hline
AA & Arzelà--Ascoli\\ \hline
AG & Auslander--Gorenstein\\ \hline
AK & Ariki--Koike\\ \hline
AK & Ax--Katz\\ \hline
AL & Atkin--Lehner\\ \hline
ALW &  Ax--Lindemann--Weierstrass\\ \hline
AO & André--Oort\\ \hline
AR & Auslander--Reiten  \\ \hline
AS & Artin--Schreier\\ \hline
AS & Atiyah--Singer\\ \hline
AS & Ax--Schanuel\\ \hline
AW & Alexander--Whitney\\ \hline
BB & Baily--Borel\\ \hline
BB & Beilinson--Bernstein\\ \hline
BBDG & Beilinson--Bernstein--Deligne--Gabber\\ \hline
BC & Banach--Colmez\\ \hline
BCH & Baker--Campbell--Hausdorff\\ \hline
BD & Breen--Deligne\\ \hline
BGG & Bernstein--Gelfand--Gelfand\\ \hline
BK & Bloch--Kato\\ \hline
BL & Bombieri--Lang\\ \hline
BL & Borel--Lebesgue\\ \hline
BM  & Blakers--Massey\\ \hline
BM & Borel--Moore\\ \hline
BM & Brauer--Manin\\ \hline
BMK & Riesz--Markov--Kakutani\\ \hline
BMS & Bhatt--Morrow--Scholze\\ \hline
BN & Browder--Novikov\\ \hline
BP & Brieskorn--Pham\\ \hline
BQ & Bloch--Quillen\\ \hline
BS & Banach--Steinhaus\\ \hline
BS & Banach--Stone\\ \hline
BS & Brumer--Stark\\ \hline
BT & Banach--Tarski\\ \hline
BT & Bruhat--Tits\\ \hline
BU & Borsuk--Ulam\\ \hline
BW & Bolzano--Weierstrass\\ \hline
BWB & Borel--Weil--Bott\\ \hline
CG & Clebsch--Gordan\\ \hline
CJ & Chen--Jiang\\ \hline
CKN & Caffarelli--Kohn--Nirenberg\\ \hline
CM & Castelnuovo--Mumford\\ \hline
CP &  Cauchy--Pompeiu\\ \hline
CR & Cauchy--Riemann\\ \hline
CS & Cappell--Shaneson\\ \hline
CS & Cartan--Serre\\ \hline
CS & Cauchy--Schwarz\\ \hline
CS & Clausen-Scholze\\ \hline
CS & Cotlar--Stein\\ \hline
CV & Calderon--Vaillancourt\\ \hline
CW & Chevalley--Warning\\ \hline
CY & Calabi--Yau\\ \hline
DB & Deligne--Beilinson\\ \hline
DK & Dold--Kan\\ \hline
DL & Deligne--Lusztig\\ \hline
DM & Deligne--Mumford\\ \hline
DM & Dieudonné--Manin\\ \hline
DP & De Concini--Procesi\\ \hline
DP & Dold--Puppe\\ \hline
DS & Deligne--Serre \\ \hline
DT & Dold--Thom\\ \hline
DT & Donaldson--Thomas\\ \hline
DW & De Rham--Weil\\ \hline
DW & Dowling--Wilson\\ \hline
EH & Eckmann--Hilton\\ \hline
EK & Enriques--Kodaira\\ \hline
EL & Euler--Lagrange\\ \hline
EM & Eilenberg--MacLane\\ \hline
ES & Eichler--Shimura\\ \hline
ES & Eilenberg--Steenrod\\ \hline
EW & Eilenberg--Watts\\ \hline
EZ & Eilenberg--Zilber\\ \hline
FF & Fargues--Fontaine\\ \hline
FK & Feynman--Kac\\ \hline
FL & Fontaine--Laffaille\\ \hline
FM & Fontaine--Mazur\\ \hline
FM & Fourier--Mellin\\ \hline
FM & Fourier--Mukai\\ \hline
FM & Freyd--Mitchell\\ \hline
FS & Fargues--Scholze\\ \hline
FS & Fourier--Sato\\ \hline
FS & Frobenius--Schur\\ \hline
FT & Farrell--Tate\\ \hline
FT & Feit--Thompson\\ \hline
FU & Fréchet--Urysohn\\ \hline
FW & Fontaine--Winterberger\\ \hline
GGP & Gan--Gross--Prasad\\ \hline
GL & Green--Lazarsfeld\\ \hline
GM & Gauss--Manin\\ \hline
GM & Goresky--MacPherson\\ \hline
GP & Gross--Prasad\\ \hline
GP & Gross--Prasad\\ \hline
GS & Garcia--Sankaran\\ \hline
GS & Golod--Shafarevich\\ \hline
GS & Gram--Schmidt\\ \hline
GV & Gopakumar--Vafa\\ \hline
GW & Grunwald--Wang\\ \hline
GZ & Gross--Zagier\\ \hline
HB & Hahn--Banach\\ \hline
HB & Heine--Borel\\ \hline
HJ & Hamilton--Jacobi\\ \hline
HL & Hardy--Littlewood\\ \hline
HLS & Hardy--Littlewood--Sobolev\\ \hline
HM & Hasse--Minkowski\\ \hline
HN & Harder--Narasimhan\\ \hline
HR & Hodge--Riemann\\ \hline
HT & Hodge--Tate\\ \hline
HW & Hasse--Weil\\ \hline
HZ & Hirzebruch--Zagier\\ \hline
JH & Jordan--Hölder\\ \hline
JM & Jacobson--Morozov\\ \hline
KA & Krull--Akizuki\\ \hline
KAM & Kolmogorov--Arnold--Moser\\ \hline
KL & Kazhdan--Lusztig \\ \hline
KM & Kac--Moody\\ \hline
KR & Kudla--Rapoport\\ \hline
KS & Kashiwara--Schapira\\ \hline
KS & Kelvin--Stokes\\ \hline
KS & Kirby--Siebenmann\\ \hline
KS & Kodaira--Spencer\\ \hline
KS & Krull--Schmidt\\ \hline
KW & Kronecker--Weber\\ \hline
LH & Leray--Hirsch\\ \hline
LK & Langlands--Kottwitz\\ \hline
LM & Levi--Malcev\\ \hline
LO & Littlewood--Offord\\ \hline
LR & Langlands--Rapoport\\ \hline
LT & Langlands--Tunnell\\ \hline
LT & Lubin--Tate\\ \hline
LV & Lawrence--Venkatesh\\ \hline
LZ & Liu--Zheng\\ \hline
LZ & Lu--Zheng\\ \hline
MA & Monge--Ampère\\ \hline
ML & Mordell--Lang\\ \hline
MM & Manin--Mumford\\ \hline
MN & Milnor--Novikov\\ \hline
MP & Moore--Postnikov\\ \hline
MS & Merkurjev--Suslin\\ \hline
MT & Mumford--Tate\\ \hline
MV & Mayer--Vietoris\\ \hline
NP & Newton--Puiseux\\ \hline
NS & Navier-Stokes\\ \hline
NS & Nielsen--Schreier\\ \hline
NS & Nikolov--Segal\\ \hline
NU & Neukirch--Uchida\\ \hline
NU & Neukirch--Uchida\\ \hline
PH & Poincaré--Hopf\\ \hline
PL & Phragmén--Lindelöf\\ \hline
PL & Poincare--Lefschetz\\ \hline
PT & Pontryagin--Thom\\ \hline
PV & Poincaré--Verdier\\ \hline
PW & Peter--Weyl\\ \hline
RH & Riemann--Hurwitz\\ \hline
RHW & Rota--Heron--Welsh\\ \hline
RM & Riesz--Markov\\ \hline
RMK & Riesz--Markov--Kakutani\\ \hline
RS & Rankin--Selberg\\ \hline
RS & Riemann--Stieltjes\\ \hline
RT & Reshetikhin--Turaev\\ \hline
RT & Riesz--Thorin\\ \hline
RZ & Rapoport--Zink\\ \hline
SB & Severi--Brauer\\ \hline
SN & Skolem--Noether\\ \hline
SS & Sobolev-Slobodeckij\\ \hline
SS & Stanley--Stembridge\\ \hline
ST & Serre-Tate\\ \hline
SW & Schur--Weyl\\ \hline
SW & Shareshian--Wachs\\ \hline
SW & Siegel--Weil\\ \hline
SW & Spanier--Whitehead\\ \hline
SW & Stiefel--Whitney\\ \hline
SZ & Schur--Zassenhaus\\ \hline
SČ & Stone--Čech\\ \hline
TN & Tate--Nakayama\\ \hline
TS & Thom--Sebastiani\\ \hline
TT & Tomita--Takesaki\\ \hline
TW & Taylor--Wiles\\ \hline
WD & Weil--Deligne\\ \hline
WW & Wigner--Weyl\\ \hline
ZP & Zilber--Pink\\ \hline
ZR & Zariski--Riemann\\ \hline
 & \\ \hline
 & \\ \hline
 & \\ \hline
\end{longtable}

\section{Subjects related}

\begin{longtable}{l|l}
\hline
AG & analytic geometry \\ \hline
AG & algebraic geometry \\ \hline
AG & arithmetic geometry \\ \hline
CFT  & continuous Fourier transform\\ \hline
CFT & class field theory\\ \hline
CFT & conformal field theory\\ \hline
DDG & discrete differential geometry\\ \hline
DG & differential geometry\\ \hline
DG & differential graded\\ \hline
DGS & differential graded sheaf\\ \hline
GMT  & geometrical measure theory\\ \hline
LA & linear algebra  \\ \hline
RT & representation theory\\ \hline
 & \\ \hline
 & \\ \hline
 & \\ \hline
 & \\ \hline
\end{longtable}

\begin{longtable}{l|l}
\hline
LLC & local langlands correspondence \\ \hline
GLC & global langlands correspondence  \\ \hline
MMP & minimal model program\\ \hline
 & \\ \hline
 & \\ \hline
 & \\ \hline
 & \\ \hline
\end{longtable}

\section{Geometrical objects}
\begin{longtable}{l|l}
\hline
EC & elliptic curve \\ \hline
MF & modular form  \\ \hline
TVS & topological vector space\\ \hline
LCTVS &  locally convex topological vector spaces\\ \hline
LF & limit of Fréchet spaces\\ \hline
IC & intersection complex\\ \hline
mHs & mixed Hodge structure\\ \hline
\end{longtable}

\section{Other math stuffs}
\begin{longtable}{l|l}
\hline
SC & Schanuel Conjecture \\ \hline
sc & supercuspidal\\ \hline
sc & superconformal\\ \hline
sc & semicontinuity\\ \hline
sc & simply connected\\ \hline
ss & supersingular\\ \hline
ss & semisimple\\ \hline
ss & semistable\\ \hline
ss & semistandard\\ \hline
FT & Fourier transform\\ \hline
HT & Hilbert transform\\ \hline
psh & plurisubharmonic\\ \hline
spsh & strictly plurisubharmonic\\ \hline
pscv & pseudoconvex\\ \hline
spcv & strictly pseudoconvex\\ \hline 
CS & classical symbol\\ \hline
CS & computer science\\ \hline
CM & complex multiplication\\ \hline
Bl  & block\\ \hline
Bl  & blow up\\ \hline
SYT  & standard Young diagram\\ \hline
ES  & Euler system\\ \hline
PD & Poincaré duality\\ \hline
PL & piecewise linear\\ \hline
SNC & single normal crossing\\ \hline
\end{longtable}

\section{Other non-math stuffs}
\begin{longtable}{l|l}
\hline
CSG & Constructive solid geometry\\ \hline
\end{longtable}


\section{Universities}
\begin{longtable}{l|l}
\hline
HU & Humboldt-Universität zu Berlin \\ \hline
TU & Technische Universität Berlin  \\ \hline
FU & Freie Universität Berlin\\ \hline
BMS & Berlin Mathematical School\\ \hline
 & \\ \hline
 & \\ \hline
 & \\ \hline
\end{longtable}

Berlin:
\begin{longtable}{l|l}
\hline
RTG & Research Training Groups \\ \hline
IMPRS & International Max Planck Research Schools  \\ \hline
WIAS & Weierstrass Institute for Applied Analysis and Stochastics\\ \hline
 & \\ \hline
 & \\ \hline
 & \\ \hline
 & \\ \hline
\end{longtable}

\end{document}