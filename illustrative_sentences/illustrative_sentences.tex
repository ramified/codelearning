
\documentclass[11pt]{amsart}

%\usepackage{color,graphicx}
%\usepackage{mathrsfs,amsbsy}
\usepackage{ctex}
\usepackage{CJKfntef}
\usepackage{amssymb}
\usepackage{amsmath}
\usepackage{amsfonts}
\usepackage{graphicx}
\usepackage{amsthm}
\usepackage{enumerate}
\usepackage[mathscr]{eucal}
\usepackage{mathrsfs}
\usepackage{verbatim}
\usepackage{listings}
\usepackage{xcolor}
\usepackage{url}
\usepackage{hyperref}
\usepackage{fancybox}

%\usepackage[notcite,notref]{showkeys}

% showkeys  make label explicit on the paper



%python settings
\definecolor{halfgray}{gray}{0.55}
\definecolor{ipython_frame}{RGB}{207, 207, 207}
\definecolor{ipython_bg}{RGB}{247, 247, 247}
\definecolor{ipython_red}{RGB}{186, 33, 33}
\definecolor{ipython_green}{RGB}{0, 128, 0}
\definecolor{ipython_cyan}{RGB}{64, 128, 128}
\definecolor{ipython_purple}{RGB}{170, 34, 255}

\usepackage{listings}
\lstset{
	breaklines=true,
	%
	extendedchars=true,
	literate=
	{á}{{\'a}}1 {é}{{\'e}}1 {í}{{\'i}}1 {ó}{{\'o}}1 {ú}{{\'u}}1
	{Á}{{\'A}}1 {É}{{\'E}}1 {Í}{{\'I}}1 {Ó}{{\'O}}1 {Ú}{{\'U}}1
	{à}{{\`a}}1 {è}{{\`e}}1 {ì}{{\`i}}1 {ò}{{\`o}}1 {ù}{{\`u}}1
	{À}{{\`A}}1 {È}{{\'E}}1 {Ì}{{\`I}}1 {Ò}{{\`O}}1 {Ù}{{\`U}}1
	{ä}{{\"a}}1 {ë}{{\"e}}1 {ï}{{\"i}}1 {ö}{{\"o}}1 {ü}{{\"u}}1
	{Ä}{{\"A}}1 {Ë}{{\"E}}1 {Ï}{{\"I}}1 {Ö}{{\"O}}1 {Ü}{{\"U}}1
	{â}{{\^a}}1 {ê}{{\^e}}1 {î}{{\^i}}1 {ô}{{\^o}}1 {û}{{\^u}}1
	{Â}{{\^A}}1 {Ê}{{\^E}}1 {Î}{{\^I}}1 {Ô}{{\^O}}1 {Û}{{\^U}}1
	{œ}{{\oe}}1 {Œ}{{\OE}}1 {æ}{{\ae}}1 {Æ}{{\AE}}1 {ß}{{\ss}}1
	{ç}{{\c c}}1 {Ç}{{\c C}}1 {ø}{{\o}}1 {å}{{\r a}}1 {Å}{{\r A}}1
	{€}{{\EUR}}1 {£}{{\pounds}}1
}

%%
%% Python definition (c) 1998 Michael Weber
%% Additional definitions (2013) Alexis Dimitriadis
%% modified by me (should not have empty lines)
%%
\lstdefinelanguage{iPython}{
	morekeywords={access,and,break,class,continue,def,del,elif,else,except,exec,finally,for,from,global,if,import,in,is,lambda,not,or,pass,print,raise,return,try,while},%
	%
	% Built-ins
	morekeywords=[2]{abs,all,any,basestring,bin,bool,bytearray,callable,chr,classmethod,cmp,compile,complex,delattr,dict,dir,divmod,enumerate,eval,execfile,file,filter,float,format,frozenset,getattr,globals,hasattr,hash,help,hex,id,input,int,isinstance,issubclass,iter,len,list,locals,long,map,max,memoryview,min,next,object,oct,open,ord,pow,property,range,raw_input,reduce,reload,repr,reversed,round,set,setattr,slice,sorted,staticmethod,str,sum,super,tuple,type,unichr,unicode,vars,xrange,zip,apply,buffer,coerce,intern},%
	%
	sensitive=true,%
	morecomment=[l]\#,%
	morestring=[b]',%
	morestring=[b]",%
	%
	morestring=[s]{'''}{'''},% used for documentation text (mulitiline strings)
	morestring=[s]{"""}{"""},% added by Philipp Matthias Hahn
	%
	morestring=[s]{r'}{'},% `raw' strings
	morestring=[s]{r"}{"},%
	morestring=[s]{r'''}{'''},%
	morestring=[s]{r"""}{"""},%
	morestring=[s]{u'}{'},% unicode strings
	morestring=[s]{u"}{"},%
	morestring=[s]{u'''}{'''},%
	morestring=[s]{u"""}{"""},%
	%
	% {replace}{replacement}{lenght of replace}
	% *{-}{-}{1} will not replace in comments and so on
	literate=
	*{+}{{{\color{ipython_purple}+}}}1
	{-}{{{\color{ipython_purple}-}}}1
	{*}{{{\color{ipython_purple}$^\ast$}}}1
	{/}{{{\color{ipython_purple}/}}}1
	{^}{{{\color{ipython_purple}\^{}}}}1
	{?}{{{\color{ipython_purple}?}}}1
	{!}{{{\color{ipython_purple}!}}}1
	{\%}{{{\color{ipython_purple}\%}}}1
	{<}{{{\color{ipython_purple}<}}}1
	{>}{{{\color{ipython_purple}>}}}1
	{|}{{{\color{ipython_purple}|}}}1
	{\&}{{{\color{ipython_purple}\&}}}1
	{~}{{{\color{ipython_purple}~}}}1
	%
	{==}{{{\color{ipython_purple}==}}}2
	{<=}{{{\color{ipython_purple}<=}}}2
	{>=}{{{\color{ipython_purple}>=}}}2
	%
	{+=}{{{+=}}}2
	{-=}{{{-=}}}2
	{*=}{{{$^\ast$=}}}2
	{/=}{{{/=}}}2,
	%
	literate=
	{á}{{\'a}}1 {é}{{\'e}}1 {í}{{\'i}}1 {ó}{{\'o}}1 {ú}{{\'u}}1
	{Á}{{\'A}}1 {É}{{\'E}}1 {Í}{{\'I}}1 {Ó}{{\'O}}1 {Ú}{{\'U}}1
	{à}{{\`a}}1 {è}{{\`e}}1 {ì}{{\`i}}1 {ò}{{\`o}}1 {ù}{{\`u}}1
	{À}{{\`A}}1 {È}{{\'E}}1 {Ì}{{\`I}}1 {Ò}{{\`O}}1 {Ù}{{\`U}}1
	{ä}{{\"a}}1 {ë}{{\"e}}1 {ï}{{\"i}}1 {ö}{{\"o}}1 {ü}{{\"u}}1
	{Ä}{{\"A}}1 {Ë}{{\"E}}1 {Ï}{{\"I}}1 {Ö}{{\"O}}1 {Ü}{{\"U}}1
	{â}{{\^a}}1 {ê}{{\^e}}1 {î}{{\^i}}1 {ô}{{\^o}}1 {û}{{\^u}}1
	{Â}{{\^A}}1 {Ê}{{\^E}}1 {Î}{{\^I}}1 {Ô}{{\^O}}1 {Û}{{\^U}}1
	{œ}{{\oe}}1 {Œ}{{\OE}}1 {æ}{{\ae}}1 {Æ}{{\AE}}1 {ß}{{\ss}}1
	{ç}{{\c c}}1 {Ç}{{\c C}}1 {ø}{{\o}}1 {å}{{\r a}}1 {Å}{{\r A}}1
	{€}{{\EUR}}1 {£}{{\pounds}}1,
	%
	%   identifierstyle=\color{red}\ttfamily,
	commentstyle=\color{ipython_cyan}\ttfamily,
	stringstyle=\color{ipython_red}\ttfamily,
	keepspaces=true,
	showspaces=false,
	showstringspaces=false,
	%
	rulecolor=\color{ipython_frame},
	frame=single,
	frameround={t}{t}{t}{t},
	framexleftmargin=6mm,
	numbers=left,
	numberstyle=\tiny\color{halfgray},
	%
	%
	backgroundcolor=\color{ipython_bg},
	%   extendedchars=true,
	basicstyle=\scriptsize\ttfamily,
	keywordstyle=\color{ipython_green}\ttfamily,
}


\lstdefinelanguage{Sage}[]{Python}
{morekeywords={False,sage,True},sensitive=true}
\lstset{
	frame=single,%none or single
	showtabs=False,
	showspaces=False,
	showstringspaces=False,
	commentstyle={\ttfamily\color{dgreencolor}},
	keywordstyle={\ttfamily\color{keywordsage}\bfseries},
	stringstyle={\ttfamily\color{dgraycolor}\bfseries},
	language=Sage,
	basicstyle={\fontsize{10pt}{10pt}\ttfamily},
	aboveskip=0.3em,
	belowskip=0.1em,
	numbers=left,
	numberstyle=\footnotesize,
	stepnumber=5,
	breaklines=true,
	backgroundcolor=\color{backsage}
}
\definecolor{dblackcolor}{rgb}{0.0,0.0,0.0}
\definecolor{dbluecolor}{rgb}{0.01,0.02,0.7}
\definecolor{dgreencolor}{rgb}{0.2,0.4,0.0}
\definecolor{dgraycolor}{rgb}{0.30,0.3,0.30}
\newcommand{\dblue}{\color{dbluecolor}\bf}
\newcommand{\dred}{\color{dredcolor}\bf}
\newcommand{\dblack}{\color{dblackcolor}\bf}
\definecolor{backsage}{RGB}{255, 255, 229}
\definecolor{keywordsage}{RGB}{198, 93, 9}




\begin{document}
\date{}

\title
{例句}


\author{周潇翔}
\address{School of Mathematical Sciences\\
University of Science and Technology of China\\
Hefei, 230026\\ P.R. China\\} 
\email{xx352229@mail.ustc.edu.cn}





\begin{abstract}
这里收集外文例句。

Here are some example sentences for language expressions collected from articles and daily lives. 
\end{abstract}



\maketitle
%%%%%%%%%%%%%%%%%%%%%%%%%%%%%%%%%%%%%%%%%%%%%%%%%%%%%%%%%%%%%%%%%%%%%%%%%%%%%%%%%%%%%%%%%%%%%





\section{数学相关句子模板}
这里收集“啊我之前不知道咋表达,这就是我当时卡住的时候可以用的句子!”的外文句子。绝对是非原创的。
\begin{lstlisting}[numbers=left,numberstyle=\tiny,numbersep=10pt]
words for beginning (in blackboard):
Definition, Proposition, Theorem, Lemma, Corollary, Conjecture
Fact, Claim, Example, Exercise, Question, Task, Aim
Notations, Setting, Conventions, Setup, Prologue, Synopsis
Slogan, Digression, Caveat, Warning


casual expression:
whizz through
take it for granted
Many of them are better at faking it.
Students have to learn how to properly write down a proof and everything. So one has to do important pedagogical work and teach them, how to do this.
eagle eye viewer
plunge onward to
insanely general
wrap up the current discussion
One small wrinkle
get things off the ground
the third of a sequence of courses ... is teaching
I'm not sure I can jump on the train without toiling so much
lines of inquiry
Perhaps me saying something stupid will encourage an expert to weigh in
The answer is a resounding "no".
the running example
provide impetus
the interplay between


refer to:
The main references for the material covered in this section are ...
A more detailed overview of this chapter may be obtained by reading the introductions to the various sections.
More detailed treatments appear in ..., to which we will refer for proofs.
mimic the arguments for ...
We refrain from giving a more detailed introduction here and instead refer the reader to the table of contents as well as to the short introductions of the individual sections.
References for ... include the monograph ... and the more introductory account ...
The result of ... was catalyzed by reading...
follow a circuitous route
We include the ... argument here for convenience of exposition, and because the comparison with their constructions is interesting in its own right.
We refer to this theorem as the ... theorem.
All ideas from these notes were shamelessly stolen from various lecture notes in the literature. I mention a few:

Disclaimer:
minor errors and obscurities, a couple of significant lapses
I am by far not a person with serious knowledge/understanding of ..., thus in the ... I may oversimplify/overcomplicate things, be inaccurate, or even wrong, and miss subtelties.
I have actually been meaning to correct some typos for a while and so I will hopefully do all these corrections in the next few weeks.
All errors or inaccuracies are on my side.
We cannot assemble here the necessary apparatus from ... theory.
use the most naive and explicit language without having to sacrifice any essential ideas
Many of the topics in this ... have appeared elsewhere, or belong to the mathematical folklore; it should not be assumed that uncredited results are due to the author.
I felt that my students were not adequately prepared for his text, and I wrote my notes with the hope to provide this preparation.
is directed to non-experts or not-yet-experts
Though we are not aware of any reference which states these theorems in the generality which we consider, these theorems should be considered well known.
The errors that remain are of course due to the author.
This follows from the general theory, but since we have not defined ..., we are merely stating this as a fact.
All these claims are well-known; we spell out the proof only for the reader's convenience.
This is an interesting case but I've run out of time and enthusiasm for it.
We make no pretense to historical remarks.

notations:
Beginning with..., only ... are considered.
shorthand
By an aesthentically desirable abuse of notation, 
One word about the notation: for simplicity we often omit ...
For notational reasons we usually extend the ... to (negative indices) by defining ...
The term ... is a convenient, but temporary, expedient.
To save adjectives, if ... is described as ..., it is implicitly assumed to be ...
Finally, a word about terminology.
However on many occasions it is natural or customary to stay in ... mode when nonetheless one has ... in mind.
While for the most part this convention seems to work well, it occasionally leads us to make extraneous ... hypotheses in order to invoke ...
We try to flag this artifice when it occurs.
The following abuse of notation will prove handy:...
we can use the same definition in more general contexts, for example, ...
allow us to state theorems more succinctly
clutter notation
As ... will appear frequently below, we stress that ... are always ... for us.
It would be more customary to write ..., but this leads to inconsistent notation.
Recap: recapitulation
This assumption is for simplicity only
A depends on b, but we suppress this from the notation
by alliteration
(in the same
sense as for exceptional sequences)
Although we'll only use standard properties of ..., let us recall a few definitions for the comfort of the reader.
no clash of terminology
This is technically not part of the definition, but it is nice to have
everything at one place.
We will always state when we switch to ..., so if nothing else is said ... is the one that is used.
Much of this notation will also be introduced in the text, but I have
tried to collect together various definitions here, for ease of reading.
This definition may seem self-referential, but as..., it works by induction on...(with... as the base case)
Such phrases can be kept as keywords in mind and learned later.
Since this is the main subject of study of this paper, we have decided to spent this section to explicitly set-up all the notation that we will use later.

thanks:
warm hospitality
Further comments, corrections and suggestions are of course more than welcome.

example and generalization:
prototypical example
a plethora of explicit examples
By the same argument as the special case above, ...
The arguments in ... carry over mutatis mutandis.
provide a navigable route into the area with a complete and self-contained account of the case ...
eschew all generality
exploit special features to abbreviate/simplify the arguments
tinker with the examples
proceed similarly in general
In fact, we would expect similar results to hold true in the case where ...
Our arguments are not general enough to handle that case.
There is a veritable legion of examples of categories which fit this paradigm:...
specialize to our case of main interest
... generalizes ... in two directions: it works for ..., and moreover it deals with ... instead of ...
At the expense of more notation a similar result holds for the other...
The ... have their counterparts in higher
dimensions among the ...
Apart from the..., the... in the... and... situations are uniform.
The definition below is written in the context of.... In the case of..., one has to replace... with....

technical problem:
achieve some technical control
Ceci nous contraint à prouver ... par une voie détournée
It should be noted that even if one is interested only in ..., the proofs often involve more general ...
... are pivotal in this matter.
For a rigorous definitions of ... we would need the notion of ..., which would take us too far away from the subject of ...
the entire difficulty is bundled in ...
It turns out that ... is inadequate to this role, ...
It is not strictly necessary to think only in the case of ..., but it certainly allows one to ignore some technical difficulties.
address an issue
To keep the exposition brisk, we will postpone the more difficult proofs until ...
subtle distinction
the choice of ... is immaterial
for highly non-formal reasons
The following lemma summarizes the abstract setup, and isolates the key property that we need to prove in our situation.
The theory of ... involves occasionally long computations. I moved several of them to the end of .... The reader may want to do some of them as exercises without looking first at these appendices.
A plethora of techniques has been applied to ...
In spite of a similarity between the notion of ... and ... there are essential differences.
We gloss over many crucial technical details in favor of rendering a more panoramic picture; ... offer a partial remedy to these omissions.

ask for details:
I find there are a lot of cool ideas in your answers, but I would be grateful if you could be a bit more precise.
The crux of the argument lies in the proof of... that we define in more detail now.


believe something without proving them:
We believe that one could also carry out the proof of ... in the language of ..., but we  have not investigated the details.
We further suspect, but do not prove, that...
Suggestively speaking, it is as if...

digression:
We now want to describe the ... This can be done in a more straightforward way, but we prefer to include a short digression in ... theory as this allows us to mention a general fact which is in the background of a later construction anyhow.
Nevertheless, the reader can skip this digression without loss of continuity and continue with ... instead.
Choosing a linear order in presenting the foundations is no easy task.
Fall into the purview of the above theorem
In this survey we intend to give a reasonably thorough account of the recent work, though mostly without detailed proofs
let us beyond the areas where we can pretend to competence
describe sufficiently but not exhaustively the earlier work of...
The ressemblance between ... and ... will certainly not have escaped the reader's sagacity.
The case of ... is special, and is discussed separately in ...

excitement/beauty:
The literature on ... is vast and often technical, but the underlying ideas are possessing of an undeniable beauty.
The excitement provoked by ... stimulated a period of intense and widespread activity.
convey the breadth and excitement of the ideas
At first sight, ... does not seem
very natural from a ... perspective. However, at least for ..., ... is important from
... perspective, since...
carve out a hierarchy
of full subcategories of ...
colloquially speaking
Here is a concrete paraphrase of the proposition.
... is one of the pinnacles of mathematical achievement in the 20th century.
Each theory sheds its own light on these objects, and combining the various perspectives is likely to be very fruitful.
what appears to be external search turns out to be an internal search
Those to whom this seems obvious might ponder the fact that ...


connect different areas:
unveils intricate links between
reveals connections between
is akin to the phenomenon of ...
is deeply interwined with
give a bird's eye view on

waiting for classification:
equivalently(replace i.e.)
encyclopaedic knowledge of ...
In fact, the arguments used in the final part of this proof, give the following result.
Let us unravel this definition a little.(after complicated definition)
... is pivotal in this matter./ ... play a pivotal rôle. 
The arithmetic/information/structure of ... is encapsulated in ...
encapsulate a significant amount of information concerning ...
encapsulate subtlety
We point out the similarities and the differences whenever appropriate.
There is also some degree of novelty in our treatment of ...
... contiennent les outils requis pour surmonter ces difficultés.
give a sense of the directions in which the area is going.
there is nothing to be gained from specialization at this stage
take it as a little rebus
whirlwind tour of ...
be enamored with
concrete paraphrase
To avoid any vicious circles, we use the following elementary argument going back to ...
..., if the following equivalent conditions are satisfied:
Let us discuss the contents in a nutshell.
it's instructive to give here a direct argument. 
A slow start in a lecture is never an obstacle to arrive far and present involved advanced material in the end.
mutually inverse isomorphisms
needn't equal

brackets:
{}:curly set brackets
():round brackets
[]:square brackets
<>:angle brackets

beginning words: Goals, Pros and Cons
\end{lstlisting}

另外关于英语语法,可以参考\href{https://arxiv.org/ftp/arxiv/papers/1011/1011.5973.pdf}{天文物理类英文科技论文写作的常见问题}.
\section{德语日常句子}
\begin{lstlisting}[numbers=left,numberstyle=\tiny,numbersep=10pt]
Ich bin pappsatt.
gesund bleiben / gesund sein
je ... desto ...
Nichts ist umsonst.
Der Fisch war nicht üppig.
Es war ein großer Reinfall!
Wir müssen feststellen, inwiefern sich die Situation geändert hat.
ums Leben kommen
Jeder ist seines Glückes Schmied.
Unerhörtes ist geschehen.
Große und unübersehbare Arbeit steht uns bevor.
Wir müssen alle Kräfte anspannen, um ...
eine Ausgleich zwischen ... und ... herbeigeführt haben
Damit wurden die Chancen, die ..., nicht wahrgenommen.
auf ... angewiesen sein
... suchen die Wahrnehmung des Betrachters zu lenken.
hetzend zur Arbeit begeben
Es war auf Schritt und Tritt zu spüren, dass...

Anbei übersende ich Ihnen die gewünschten Unterlagen zurück.
\end{lstlisting}

\section{法语日常句子}
\begin{lstlisting}[numbers=left,numberstyle=\tiny,numbersep=10pt]
Je ne suis vraiment pas du matin.
Je ne suis pas matinal.
\end{lstlisting}
%%%%%%%%%%%%%%%%%%%%%%%%%%%%%%%%%%%%%%%%%%%%%%%%%%%%%%%%%%%%%%%%%%%%%%%%%%%%%%%%%%%%%%%%%%%%%

 
   


%%%%%%%%%%%%%%%%%%%%%%%%%%%%%%%%%%%%%%%%%%%%%%%%%%%%%%%%%%%%%%%%%%%%%%%%%%

 




%%%%%%%%%%%%%%%%%%%%%%%%%%%%%%%%%%%%%%%%%%%%%%%%%%%%%%%%%%%%%%%%%%%%%%%%%%%%%%%%%%%%%%%%%%%%%%%




\begin{thebibliography}{99}

 
%\bibitem{AF12}%
%Antunes, P., Freitas, P.: Optimal spectral rectangles and lattice ellipses. \emph{Proc. Royal Soc. London Ser. A.} \textbf{469} (2012), 20120492.


  

\end{thebibliography}


\end{document}



